\documentclass[12pt]{article}

% Load packages
\usepackage[top=1in, bottom=1in, left=1in, right=1in]{geometry}
\usepackage{graphics}
\usepackage{graphicx}
\usepackage{epsfig}
\usepackage{epsf}
\usepackage{epstopdf}
\usepackage{mathrsfs}
\usepackage{amsmath}
\usepackage{amsthm}
\usepackage{amssymb}
\usepackage{textcomp}
\usepackage[scaled=.90]{helvet} % Helvetica (currently not used)
\usepackage[]{lineno} % For line numbers
\usepackage{natbib}
\usepackage{sidecap}

%\usepackage[hidelinks]{hyperref}
%\hypersetup{
%    colorlinks,
%    citecolor=black,
%    filecolor=black,
%    linkcolor=black,
%    urlcolor=black
%}

% Define new commands and definitions
\def\ds{\displaystyle}
\def\d{\partial}
\newcommand{\rr}{\mathbb{R}}
\newcommand{\zz}{\mathbb{Z}}
\newcommand{\nn}{\mathbb{N}}
\newcommand{\ee}{\mathscr{E}}
\newcommand{\mb}{\mathbf}
\newcommand{\ol}{\overline}
\renewcommand*{\thefootnote}{\fnsymbol{footnote}}

% Text formatting
\linespread{1.5}	
%\linenumbers

\title{Direction of escape does not matter for some fish prey}

\author{Alberto Soto, William J. Stewart and Matthew J. McHenry\\
  Department of Ecology \& Evolutionary Biology,\\
  University of California, Irvine\\ \\ \\ \\}
  
%\date{}                                           % Activate to display a given date or no date


\begin{document}
\DeclareGraphicsExtensions{.eps,.pdf,.png,.gif,.jpg,.ps}

%\maketitle
{\LARGE \textbf{Supplemental Materials}}

%\tableofcontents

% --------------------------------------------------------------------------------
%\newpage
\section{Performance plateau in the slow-predator domain}
The present study included the results of numerical solutions that showed a performance plateau in the slow-predator domain, where the prey is faster than the predator. 
Here we define the boundary to this plateau, where an equivalent escape performance is achieved for a large range of escape angles $\alpha$ (Fig. 2A).
For each escape angle in this range, the prey performs equally well by not allowing the predator to approach any closer than the distance at which the prey initiates an escape.   

% Note: There shouldn't be subsections when there is only one of them.
The distance between predator and prey varies with time, as given by the following equation:
%
\begin{equation}
D^2 = ((X_0 - Ut) + Vt\cos\alpha)^2 + (Vt\sin\alpha)^2,
\label{eq:WeihsDist}
\end{equation}
%
where $U$ and $V$ are respectively the predator and prey speeds that do not vary with time, and $X_0$ is the initial position of the prey (i.e. $D^2 = X_0^2$ at $t=0$). 
As explained in our article, the distance function may be solved for the time at which the minimum distance between predator and prey is achieved.  
The minimum distance distance occurs at $t=0$ when the distance function increases monotonically with time. 
All escape angles for which this is true will yield an ideal minimum distance ($\frac{D}{X_0}=1$).
We will consider whether this is true of the following solution to Eqn. \ref{eq:WeihsDist} reported by Weihs and Webb (1984):
%
\begin{equation}
0\leq \alpha \leq \arccos (K),
\label{eq:alphabound}
\end{equation}
where $K = U/V$. For our purposes, it is helpful to formulate this inequality as follows:
%
\begin{equation}
K \leq \cos\alpha \leq 1.
\label{eq:cosinebound}
\end{equation} 
%


Toward this aim, it suffices to show that the distance function is increasing for all positive time values.
This may be achieved by proving that the derivative of Eqn. \ref{eq:WeihsDist} with respect to time is greater than or equal to zero. 
This derivative is given by the following equation:
%
\begin{equation}
\frac{\d D^2}{\d t}  = 2(t(U^2+V^2) - UX_0 + V(X_0-2tU)\cos\alpha).
\label{eq:distderivative}
\end{equation}  
%
It is helpful for our purposes to normalize this expression by the speed of the predator (e.g. $V^*=\frac{V}{U}$).
In this way, the non-dimensional form of Eqn. \ref{eq:distderivative} is given as follows:
%
\begin{equation}
\frac{\d D^{*2}}{\d t}  = \frac{2(1+V^{*2}-2V^*\cos\alpha)t}{X_0} + 2(V^*\cos\alpha -1), 
\label{eq:DLine}
\end{equation}
%

%\begin{proof}
In order to show that Eqn. \ref{eq:DLine} is nonnegative for $t\geq0$, it suffices to show that the slope is positive and the intercept (when $t=0$) is nonnegative. 
%
\begin{itemize}
\item Positive slope. \\
Because $K<1$, it follows that $V^*>1$ and Eqn. \ref{eq:cosinebound} implies that $V^*\cos\alpha \leq 1$ [TRUE??]. 
Multiplying this inequality by $-2$ and adding $1+V^2$ on both sides yields:
%
\begin{equation}
1+V^2 -2V \cos\alpha \geq 1 -2V + V^2 = (V-1)^2 > 0.
\label{eq:slope}
\end{equation}

%TODO: Restate the above equation using non-dimensional terms.  Also, perhaps we could add the last term in a new equation.  It's a bit verbose in its current form.


\item Nonnegative intercept. \\
Here we show that $V\cos \alpha -1 \geq 0.$ From Eqn. \ref{eq:cosinebound} we have:
\begin{equation}
K \leq \cos\alpha \implies 1 = VK \leq V\cos\alpha \implies 0 \leq V\cos \alpha - 1.
\label{eq:intercept}
\end{equation}  
\end{itemize}
% Perhaps this too can be non-dimensionalized and parsed into multiple (2-3) equations.

%
Eqns. \ref{eq:slope} and  \ref{eq:intercept} together show that the distance is always increasing for $0\leq \alpha \leq \arccos (K)$ and $K<1.$ 
%\end{proof}
An analogous argument applies for $-\arccos(K) \leq \alpha < 0.$ The preceding proof shows that the minimum distance occurs at $t=0$ and is given by $D^2 = X_0^2.$ This defines a performance plateau for the prey as a wide range of angles yield equally successful escapes. 



% --------------------------------------------------------------------------------
\section{Initial Lateral Displacement}
The distance function given by Eqn. \ref{eq:WeihsDist} is based on the assumption that the predator is headed directly at the initial position of the prey. 
However, previous experiments have shown that predators often fail to perfectly align their approach toward the prey.
To model this situation, here we introduce a lateral initial position to the distance function. 

\subsection{Distance function with initial lateral displacement}
This general form of the distance function is now given by the following:    
%
\begin{equation}
D^2 = ((X_0 - Ut) + Vt\cos\alpha)^2 + (Y_0 + Vt\sin\alpha)^2.
\label{eq:YDist}
\end{equation}
%
Note that the introduction of an initial lateral position ($Y_0$) allows us to rewrite the equation in polar coordinates, which simplifies our analysis below.
We can rewrite Eqn. \ref{eq:YDist} in polar coordinates $(R,\theta)$ by setting $R_0^2 = X_0^2 + Y_0^2,$ and $\theta_0 = \arctan(Y_0/X_0)$. 
Here we assume that $Y_0 \geq 0$, but the final results are presented for the more general case. 
This yields the following:
%
\begin{equation}
D_0^2 = R_0^2 + (1 + K^2) t^2 V^2 - 
 2 V t(K V t \cos \alpha  - R_0 \cos(\alpha - \theta_0) + K R_0 \cos \theta_0).
 \label{eq:Dpolar}
\end{equation}
%
To find the time at which Eqn. \ref{eq:Dpolar} is minimal, we find the roots of the derivative of Eqn. \ref{eq:Dpolar} with respect to $t$, which yields the following solution:
%
\begin{equation}
t_{\text{min}} = \ds \frac{R_0}{V} \frac{\left[K \cos \theta_0 - \cos(\alpha - \theta_0)\right ]}{1- 2K \cos \alpha + K^2}
\label{eq:tmin}
\end{equation}
%
The above is negative when $K \cos \theta_0 < \cos(\alpha - \theta_0).$ 
Rewriting this inequality gives the range of $\alpha$ for which the distance is solely increasing. Explicitly, this is given by the following:
%
\begin{equation}
\theta_0 - \arccos(K \cos \theta_0) < \alpha < \theta_0 + \arccos(K \cos \theta_0).
\label{eq:alphabound}
\end{equation}
%
For these values of $\alpha$, the minimum distance occurs at $t=0$ and is thus equal to the initial distance $R_0^2$. 
This defines a performance plateau when $K < 1.$
If we now substitute $t_{\text{min}}$ for $t$ in Eqn. \ref{eq:Dpolar}, we get the minimum distance as a function of $\alpha$ with respect to parameters $K$ and $\theta_0$. 
This is given by the following: 
%
\begin{equation}
\ol{D}^2_{\text{min}}= \ds\frac{{D}^2_{\text{min}}}{R_0^2 }=
\ds\frac{\left ( \sin(\alpha - \theta_0) + K \sin \theta_0 \right )^2}{K^2-2 K \cos \alpha +1}. 
\label{eq:Dmin_polar}
\end{equation}
%

% --------------------------------------------------------------------------------
\newpage
\subsection{Finding values of $\alpha$ that optimize the minimum distance}

To find the escape angle which yields the largest minimum distance, we solved the following equation:
%
\begin{equation}
0 = \frac{\d \ol{D}^2_{\text{min}}}{\d \alpha} = 
\frac{2(K \cos \alpha - 1)(K\cos \theta_0 - \cos(\alpha - \theta_0))(K\sin \theta_0 + \sin(\alpha -\theta_0))}
{(K^2 - 2K \cos \alpha + 1)^2}
\end{equation} 
%
The solutions to this equation are found by finding where the numerator is equal to zero which is done by considering the following three cases:
%
\begin{enumerate}
\item[]{\bf Case 1: $K \cos \alpha - 1 = 0.$} 

Solving this equation yields the following relationship:
\begin{equation}
\alpha_1 = \pm \arccos K^{-1}
\end{equation}
This solution is valid for $K\geq1,$ which indicates that prey are equally effective if escaping at an optimal angle toward the left ($\alpha>0$), or right ($\alpha<0$) of the predator's heading. 

\item[]{\bf Case 2: $K\cos \theta_0 - \cos(\alpha - \theta_0) = 0.$} 

A careful analysis is required for this case because the solution can include complex numbers for some combinations of $K$ and $\theta_0.$ 
This equation may be formulated as follows: 
%
\begin{equation}
\cos(\alpha - \theta_0) = K\cos \theta_0.
\label{eq:case2}
\end{equation}
%
This equation gives requires that values of $K$ and $\theta_0$ yield solutions which are real numbers. 
Explicitly, this condition is given by the following
%
\begin{equation}
 | K\cos \theta_0 | \leq 1
 \label{eq:boundK}
\end{equation}
%
Eqn. \ref{eq:boundK} is satisfied when $0 \leq K \leq1$. When $K > 1$, then we must have that $| \cos \theta_0| \leq 1/K.$ This leads to the following bound for $\theta_0$:
%
\begin{equation}
\arccos (K^{-1}) \leq \theta_0 \leq \arccos (-K^{-1}).
\label{eq:boundtheta0}
\end{equation}
%
Note that as the value of $K$ increases, the allowable range for $\theta_0$ decreases and approaches 90\textdegree. 
With these restrictions in mind, we proceed to solve Eqn. \ref{eq:case2}, which is given by:
%
\begin{align*}
\alpha_2 & = \theta_0 + \arccos(K \cos \theta_0), \\
\alpha_3 & = \theta_0 - \arccos(K \cos \theta_0).
\end{align*}
%
Therefore,  the following equation defines the boundaries of the performance plateau for $K<1:$ 
%
\begin{equation}
\quad |\alpha - \theta_0|  \leq    \arccos(K \cos  \theta_0).
 \end{equation}
%
For $K>1,$ the solutions $\alpha_{2,3}$ are not optimal unless $\theta_0$ simultaneously satisfies Eqn. \ref{eq:boundtheta0}. 

% I'm not sure what to make of this final statement.  Does any solution satisfy these conditions?  If not, we might want to say that we can't find any such solution. What can we conclude here about K>1?

\item[]{\bf Case 3: $K\sin \theta_0 + \sin(\alpha -\theta_0) = 0.$}

Solving this equation yields the following:
%
\begin{align*}
\alpha_4 & = \theta_0 - \arcsin(K \sin \theta_0), \\
\alpha_5 & = \pi + \theta_0 + \arcsin(K \sin \theta_0).
\end{align*}
%
For $K<1$, $\alpha_4$ is contained within the bounds defined by Eqn. \ref{eq:alphabound} and $\alpha_5$ is a local minimum. 
When $K>1$, the solutions are local minima and thus not optimal.     
\end{enumerate}
 
 % Here too I think it would be helpful to explain the meaning of these results in words.


\end{document}