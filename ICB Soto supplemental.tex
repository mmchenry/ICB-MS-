\documentclass[12pt]{article}

% Load packages
\usepackage[top=1in, bottom=1in, left=1in, right=1in]{geometry}
\usepackage{graphics}
\usepackage{graphicx}
\usepackage{epsfig}
\usepackage{epsf}
\usepackage{epstopdf}
\usepackage{mathrsfs}
\usepackage{amsmath}
\usepackage{amssymb}
\usepackage{textcomp}
\usepackage[scaled=.90]{helvet} % Helvetica (currently not used)
\usepackage[]{lineno} % For line numbers
\usepackage{natbib}
\usepackage{sidecap}

\usepackage[hidelinks]{hyperref}
\hypersetup{
    colorlinks,
    citecolor=black,
    filecolor=black,
    linkcolor=black,
    urlcolor=black
}

% Define new commands and definitions
\def\ds{\displaystyle}
\def\d{\partial}
\newcommand{\rr}{\mathbb{R}}
\newcommand{\zz}{\mathbb{Z}}
\newcommand{\nn}{\mathbb{N}}
\newcommand{\ee}{\mathscr{E}}
\newcommand{\mb}{\mathbf}
\newcommand{\ol}{\overline}
\renewcommand*{\thefootnote}{\fnsymbol{footnote}}

% Text formatting
\linespread{1.5}	
\linenumbers

\title{Escape direction does not matter for some fish prey}

\author{Alberto Soto, William J. Stewart and Matthew J. McHenry\\
  Department of Ecology \& Evolutionary Biology,\\
  University of California, Irvine\\ \\ \\ \\}
  
%\date{}                                           % Activate to display a given date or no date


\begin{document}
\DeclareGraphicsExtensions{.eps,.pdf,.png,.gif,.jpg,.ps}

\maketitle
{\Large Supplemental Materials}

\tableofcontents

% --------------------------------------------------------------------------------
\newpage
\section{Performance Plateau when $K < 1$}
In this is section, we provide details to show that when the prey is faster than the predator $(K<1)$, then there exists a performance plateau. That is, an equivalent greatest minimum distance is  achieved for a large range of escape angles $\alpha$. For escape angles in this range, the prey performs equally well.   

\subsection{Distance increases for a range of $\alpha$ values}
Here we show that the minimum distance function is not affected by a large range of escape angles. For reference, the distance function is given by the following equation:
%
\begin{equation}
D^2 = ((X_0 - Ut) + Vt\cos\alpha)^2 + (Vt\sin\alpha)^2,
\label{eq:WeihsDist}
\end{equation}
%
where $X_0$ is the starting position of the prey. Note that at $t=0$, $D^2 = X_0^2.$ 

For this case, we are interested in escape angles between the two solutions given by Weihs and Webb (1984)
%
\begin{equation}
0 \leq \alpha \leq \arccos(K).
\label{alpharange}
\end{equation}
The above leads to an important inequality given by
\begin{equation}
K \leq \cos\alpha \leq 1.
\label{cosinebound}
\end{equation} 
This inequality plays a crucial role in the following analysis.      
%
To show that the minimum distance does not change for this range of escape angles and $K<1$, it suffices to show that the distance function is always increasing. That is, we want to show that the derivative of Eqn. \ref{eq:WeihsDist} with respect to time is greater than or equal to zero. 

Taking the derivative with respect to time yields
%
\begin{equation}
\frac{\d D^2}{\d t}  = 2(t(U^2+V^2) - UX_0 + V(X_0-2tU)\cos\alpha)
\label{distderivative}
\end{equation}  
%
Let $U=1$. Since $K = U/V,$ the assumption on the predator's speed leads to $V > 1.$ 
Then equation \eqref{distderivative} becomes
%
\begin{equation}
\frac{\d D^2}{\d t}  = 2(1+V^2-2V\cos\alpha)t + 2X_0(V\cos\alpha -1), 
\label{eq:DLine}
\end{equation}
%
which is a linear function in time. To show Eqn. \ref{eq:DLine} is nonnegative for $t\geq0$, it suffices to show that the slope is positive and the intercept (when $t=0$) is nonnegative. 
%
\begin{itemize}
\item Positive slope; \\
Eqn. \ref{cosinebound} implies $V\cos\alpha \leq V$. Multiplying by $-2$ and adding $1+V^2$ on both sides yields  
\begin{equation}
1+V^2 -2V \cos\alpha \geq 1 -2V + V^2 = (V-1)^2 > 0.
\label{eq:slope}
\end{equation}

\item Nonnegative intercept; \\
Here we simply need to show that $V\cos \alpha -1 \geq 0.$ From Eqn. \ref{cosinebound} we have:
\begin{equation}
K \leq \cos\alpha \implies 1 = VK \leq V\cos\alpha \implies 0 \leq V\cos \alpha - 1.
\label{eq:intercept}
\end{equation}  
\end{itemize}
%
Eqns. \ref{eq:slope} and  \ref{eq:intercept} together show that the distance is always increasing for $0\leq \alpha \leq \arccos (K)$ and $K<1.$ An analogous argument applies for $-\arccos(K) \leq \alpha < 0.$ Thus, the minimum occurs at $t=0$ and is $D^2 = X_0^2.$ This defines a performance plateau for the prey as a wide range of angles yield equally successful ecapes. 



% --------------------------------------------------------------------------------
\section{Initial Lateral Displacement}
The distance function given by Eqn. \ref{eq:WeihsDist} is based on the assumption that a predator is heading in the direction of the prey. To model a predator that does not perfectly align its motion with the prey's direction, as is presented in the main text for a robot predator, we introduce a lateral initial displacement. 

\subsection{Distance function with initial lateral displacement}
The distance function is now given by    
%
\begin{equation}
D^2 = ((X_0 - Ut) + Vt\cos\alpha)^2 + (Y_0 + Vt\sin\alpha)^2.
\label{eq:YDist}
\end{equation}
%
In the following sections, we detail the steps for finding the optimal escape angle. Note that the introduction of $Y_0$ in the distance function allows us to rewrite the equation in polar coordinates. 

\subsection{Minimum distance in polar coordinates}
To simplify the forthcoming analysis, we rewrite Eqn. \ref{eq:YDist} in polar coordinates $(R,\theta)$. We do this by setting $R_0^2 = X_0^2 + Y_0^2,$ and $\theta_0 = \arctan(Y_0/X_0)$\footnote{For this analysis, we assume $Y_0 \geq 0$, but the final results are presented for the more general case.}. This yields 
%
\begin{equation}
D_0^2 = R_0^2 + (1 + K^2) t^2 V^2 - 
 2 V t(K V t \cos \alpha  - R_0 \cos(\alpha - \theta_0) + K R_0 \cos \theta_0) 
 \label{eq:Dpolar}
\end{equation}
%
To find the time at which Eqn. \ref{eq:Dpolar} is minimized, we find the roots of the derivative of Eqn. \ref{eq:Dpolar} with respect to $t$ which yields
%
\begin{equation}
t_{\text{min}} = \ds \frac{R_0}{V} \frac{\left[K \cos \theta_0 - \cos(\alpha - \theta_0)\right ]}{1- 2K \cos \alpha + K^2}
\label{eq:tmin}
\end{equation}
%
The above is negative when $K \cos \theta_0 < \cos(\alpha - \theta_0).$ Rewriting this inequality gives the range of $\alpha$ for which the distance is solely increasing. Explicitly, this is given by  
%
\begin{equation}
\theta_0 - \arccos(K \cos \theta_0) < \alpha < \theta_0 + \arccos(K \cos \theta_0)
\label{eq:alphabound}
\end{equation}
%
For these values of $\alpha$, the minimum distance occurs at $t=0$ and is thus equal to the initial distance $R_0^2$. This defines a performance plateau when $K < 1.$

If we now substitute $t_{\text{min}}$ for $t$ in Eqn. \ref{eq:Dpolar}, we get the minimum distance as a function of $\alpha$ with parameters $K$ and $\theta_0$. This is given by the following: 
%
\begin{equation}
\ol{D}^2_{\text{min}}= \ds\frac{{D}^2_{\text{min}}}{R_0^2 }=
\ds\frac{\left ( \sin(\alpha - \theta_0) + K \sin \theta_0 \right )^2}{K^2-2 K \cos \alpha +1} 
\label{eq:Dmin_polar}
\end{equation}
%

% --------------------------------------------------------------------------------
\newpage
\subsection{Greatest minimum distance: solving for $\alpha$}
To find the escape angle which yields the largest minimum distance, we solve the following equation
%
\begin{equation}
0 = \frac{\d \ol{D}^2_{\text{min}}}{\d \alpha} = 
\frac{2(K \cos \alpha - 1)(K\cos \theta_0 - \cos(\alpha - \theta_0))(K\sin \theta_0 + \sin(\alpha -\theta_0))}
{(K^2 - 2K \cos \alpha + 1)^2}
\end{equation} 
%
The solutions to this equation are found by finding where the numerator is equal to zero which is done by considering the following three cases:
%
\begin{enumerate}
\item[]{\bf Case 1:} Solving the equation $K \cos \alpha - 1 = 0$ yields
\begin{equation*}
\alpha_1 = \pm \arccos K^{-1}
\end{equation*}
This solution is valid for $K\geq1.$

\item[]{\bf Case 2:} For the equation $K\cos \theta_0 - \cos(\alpha - \theta_0) = 0$ a more careful analysis (detailed below) is required since the solution can be a complex number for some combinations of $K$ and $\theta_0.$ Rewriting the equation yields 
%
\begin{equation}
\cos(\alpha - \theta_0) = K\cos \theta_0.
\label{eq:case2}
\end{equation}
%
This equation gives the conditions that $K$ and $\theta_0$ must satisfy so that solutions are real--valued. Explicitly, the condition is given by the following
%
\begin{equation}
 | K\cos \theta_0 | \leq 1
 \label{eq:boundK}
\end{equation}
%
When $K$ is between 0 and 1, Eqn. \ref{eq:boundK} is always satisfied. If, on the other hand, $K > 1$, then we must have that $| \cos \theta_0| \leq 1/K.$ This leads to the following bound for $\theta_0$:
%
\begin{equation}
\arccos (K^{-1}) \leq \theta_0 \leq \arccos (-K^{-1})
\label{eq:boundtheta0}
\end{equation}

Keeping these restrictions in mind we proceed to solve Eqn. \ref{eq:case2}. The solutions are given by
\begin{align*}
\alpha_2 & = \theta_0 + \arccos(K \cos \theta_0) \\
\alpha_3 & = \theta_0 - \arccos(K \cos \theta_0)
\end{align*}
For $K<1,$ these solutions define the boundaries of the performance plateau. For $K>1$ the solutions are not optimal unless $\theta_0$ satisfies Eqn. \ref{eq:boundtheta0}.

\item[]{\bf Case 3:} Solving the equation $K\sin \theta_0 + \sin(\alpha -\theta_0) = 0$ yields
\begin{align*}
\alpha_4 & = \theta_0 - \arcsin(K \sin \theta_0) \\
\alpha_5 & = \pi + \theta_0 + \arcsin(K \sin \theta_0)
\end{align*}
For $K<1$, $\alpha_4$ is contained within the bounds defined by Eqn. \ref{eq:alphabound} and $\alpha_5$ is a local minimum. When $K>1$, the both solutions are local minimums and thus not optimal.     
\end{enumerate}



\end{document}