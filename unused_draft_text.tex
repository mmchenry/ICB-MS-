\documentclass[12pt]{article}

% Load packages
\usepackage[top=1in, bottom=1in, left=1in, right=1in]{geometry}
\usepackage{graphics}
\usepackage{graphicx}
\usepackage{epsfig}
\usepackage{epsf}
\usepackage{epstopdf}
\usepackage{mathrsfs}
\usepackage{amsmath}
\usepackage{amssymb}
\usepackage{textcomp}
\usepackage[scaled=.90]{helvet} % Helvetica (currently not used)
\usepackage[]{lineno} % For line numbers
\usepackage{natbib}

% Define new commands and definitions
\def\ds{\displaystyle}
\def\d{\partial}
\newcommand{\rr}{\mathbb{R}}
\newcommand{\zz}{\mathbb{Z}}
\newcommand{\nn}{\mathbb{N}}
\newcommand{\ee}{\mathscr{E}}
\newcommand{\mb}{\mathbf}
\newcommand{\ol}{\overline}

% Text formatting
\linespread 2	
\linenumbers

\title{Escape direction does not matter for some fish prey}
\author{Matthew J. McHenry, Alberto Soto}
%\date{}                                           % Activate to display a given date or no date


\begin{document}
\DeclareGraphicsExtensions{.eps,.pdf,.png,.gif,.jpg,.ps}

\maketitle
{\Large Supplementary Materials}

\tableofcontents
% --------------------------------------------------------------------------------
\section{Intro to Discussion?}


There are a number of features of piscivorous interactions which have yet to receive the attention of theoreticians and therefore have unclear strategic implications. For example, pursuit games often assume that both pursuer and evader possess perfect information, which is analogous to formulating strategy in a chess match \citep{Salen:2004wp}. In contrast, predator-prey interactions may proceed more like a game of cards, where both players possess partial information about the state and capabilities of their opponent. Therefore, sensory systems may constrain the ability to conform to an optimal strategy or different optima may be possible for a player operating with imperfect information. 

Mechanical constraints may also inform the strategy of predators and prey. The performance of motor systems is constrained by neuro--muscular physiology and, in the case of fish, hydrodynamics. The mechanics of suction feeding offers some strategic constraints on the predator. The impulsive burst of low pressure that a predator generates for suction feeding is limited in duration to tens of milliseconds  by the finite expansion permitted by the buccal cavity \citep{Wainwright:2001ufa}. In addition, the pressure gradient that captures prey is only effective in a small region (about one-half of the gape diameter) in front of the mouth \citep{Day:2005p5856}. These spatial and temporal restrictions appear to necessitate high accuracy in strike targeting and may explain why a large diversity of predators slowly glide, or brake, on their approach toward prey \citep{Higham:2005iu, Higham:2007go}. 


Prey fish will generally initiate a high-speed escape response if they detect an approaching predator. This fast start€™ escape consists of curling the body into a ‘C’ shape (Stage 1) and then unfurling (Stage 2) to initiate high-speed swimming \citep{D:1973up}. The neurophysiology of the fast start has offered an prolific model on vertebrate motor control (CITE) and some investigators have considered the strategic implications of fast-start kinematics \citep[reviewed by][]{Domenici:2011tv, Domenici:2011vl}. However, it remains unclear how the direction of a fast start affects a prey's probability of surviving an encounter with a predatory fish. 




% --------------------------------------------------------------------------------
\section{[Text from earlier versions]}


This effect may be understood by considering the timing of the minimum distance formulated by Weihs and Webb (1984). 

For $K<1$ the prey is faster than the predator. In this case, intuition leads us to accept that an escape directly away from the predator ($\alpha =0$) is the best solution. What is not so obvious, is that there exists a range of possible escape angles for which the minimum distance function does not change.That is, there is no single optimal escape angle. Thus, a prey can escape along any direction within this range with no penalty.

We begin by noting that at $t=0$, $D^2 = X_0^2.$ For this case, we are interested in escape angles between the two optimal solutions given by Weihs and Webb
%
\begin{equation}
0 \leq \alpha \leq \arccos(K).
\label{anglerange}
\end{equation}
The above inequality leads to 
\begin{equation}
K \leq \cos\alpha \leq 1.
\label{cosinebound}
\end{equation}      
%
To show that the minimum distance does not change for this range of escape angles and $K<1$, it suffices to show that the distance function is always increasing. That is, we want to show that the derivative with respect to time is greater than or equal to zero. 

Taking the derivative of equation \eqref{dist} with respect to time yields
%
\begin{equation}
\frac{\d D^2}{\d t}  = 2(t(U^2+V^2) - UX_0 + V(X_0-2tU)\cos\alpha)
\label{distderivative}
\end{equation}  
%
Let $X_0=1$ and $U=1$. Since $K = U/V,$ the assumption on the predator's speed leads to $V > 1.$ 
Then equation \eqref{distderivative} becomes
%
\begin{equation}
\frac{\d D^2}{\d t}  = 2(t(1+V^2) +V\cos\alpha - (2tV + 1))
\label{distderivative2}
\end{equation}
%
Using the condition on $\cos\alpha$ given in equation \eqref{cosinebound} leads to

\begin{align*}
\frac{\d D^2}{\d t}  & = 2(t(1+V^2) +V\cos\alpha - (2tV + 1)) \\
& \geq 2(t(1+V^2) +VK - (2tV + 1)) \\
& = 2(t(1+V^2) +1 - (2tV + 1)) \\
& = 2(t(V^2- 2V + 1)) \\
& =2(t(V-1)^2)\\
& \geq 0.
\end{align*}2

This shows that the distance function is always in increasing. Thus, the minimum occurs at $t=0$ and is $D^2 = X_0^2 = 1.$

%-------Older text on y-displacement----------------------------- 
\section{[Earlier version]--Distance Function with Initial y--displacement}
The distance function (Eqn. \eqref{dist}) is based on the assumption that a predator is heading in the direction of the prey. A slight modification of the distance function is necessary to model a predator that does not perfectly align its motion with the prey's direction. The modified distance function is given by    
%
\begin{equation}
\ol D^2 = ((X_0 - Ut) + Vt\cos\alpha)^2 + (Y_0 + Vt\sin\alpha)^2
\label{eq:YDist}
\end{equation}
%
A similar analysis as above leads us to the minimum distance function
%
\begin{equation}
\ol D^2_{min}=\frac{(K Y_0+X_0 \sin \alpha -Y_0\cos \alpha)^2}{K^2-2 K \cos \alpha +1} 
\label{newDmin2}
\end{equation}
%
Note that this function now depends on the parameters $K$, $X_0$, and $Y_0$, whereas in the simpler case the only parameter was $K$. To maximize $\ol D^2_{min}$ as a function of $\alpha$, we solve
% 
\begin{equation}
0 = \frac{\d \ol D^2_{min}}{\d \alpha} = \frac{2(K \cos \alpha - 1)
	(K Y_0 - Y_0 \cos \alpha + X_0 \sin\alpha)(K X_0 - X_0 \cos \alpha - Y_0 \sin \alpha)}
	{(K^2 - 2K \cos \alpha + 1)^2}
\label{newDmin_dalpha}	
\end{equation}
%
We need only to consider the numerator as the denominator is zero only when $K = \cos \alpha = 1$ and represents the case in which the predator and prey travel at the same speed and in the same direction. There are three cases to consider and each may lead to several possible solutions.
% 
\subsection{Case 1}
We first consider $K \cos \alpha - 1=0$. The solution in this case is $\alpha = \arccos(1/K)$ and is only valid for $K\geq 1$. This is also one of the solutions found by Weihs and Webb.
%
\subsection{Case 2}
Next we consider the equation 
%
\begin{equation}
K Y_0 - Y_0 \cos \alpha + X_0 \sin\alpha = 0.
\end{equation}
%
There are four possible solutions 
%
\begin{equation}
\alpha = \pm \arccos \left( \frac{K Y_0^2 \pm X_0 
\ds\sqrt{Y_0^2(1- K^2) + X_0^2 }}{X_0^2+Y_0^2} \right )
\end{equation}
%	
\subsection{Case 3}
For the final case we consider the equation 
%
\begin{equation}
K X_0 - X_0 \cos \alpha - Y_0 \sin\alpha = 0.
\end{equation}
%
Again, there are four possible solutions
%
\begin{equation}
\alpha = \pm \arccos \left( \frac{K X_0^2 \pm Y_0 
\ds \sqrt{X_0^2(1-K^2) +Y_0^2}}{X_0^2+Y_0^2} \right )
\end{equation}
%

The solutions found above for Case 2 and Case 3 can be organized into those which yield positive escape angles ($+ \arccos$) and those which yield negative escape angles. For Case 2, the positive optimal escape angles are 
%
\begin{align}
\alpha_{1} =  \arccos \left( \frac{K Y_0^2 +X_0 
\ds\sqrt{Y_0^2(1- K^2) + X_0^2 }}{X_0^2+Y_0^2} \right )
\tag{15a} \label{root_a} \\
\alpha_{2} = \arccos \left(\frac{K Y_0^2 - X_0 
\ds\sqrt{Y_0^2(1- K^2) + X_0^2 }}{X_0^2+Y_0^2} \right )
\tag{15b} \label{root_b}
\end{align}
%
Both $\alpha_1$ and $\alpha_2$ are valid for values of $K$ that satisfy 
$0< K \leq \ds \frac{\sqrt{X_0^2 + Y_0^2}}{Y_0}.$




We were then able to normalize the the initial position with respect to the distance from the predator by converting this equation to polar coordinates $(R, \theta)$. This was achieved by setting $R_0^2 = X_0^2 + Y_0^2$ and $\theta_0 = \arctan(Y_0/X_0)$, which yielded the following:
%
\begin{equation}
\ol{D}^2_{min}= \ds\frac{{D}^2_{min}}{R_0^2 }=
\ds\frac{\left ( \sin(\alpha - \theta_0) + K \sin \theta_0 \right )^2}{K^2-2 K \cos \alpha +1} 
\label{Dmin_polar}
\end{equation}
%
%TODO:Check with Alberto about my changes to the definition of normalized Dmin
As for Eqn. \ref{eq37weihs}, we found the optimal escape angle $\alpha$ by finding the conditions where the second derivative of the minimum distance with respect to alpha was equal to zero:
%
\begin{equation}
0 = \frac{\d \ol{D}^2_{min}}{\d \alpha} = 
\frac{2(K \cos \alpha - 1)(K\cos \theta_0 - \cos(\alpha - \theta_0))(K\sin \theta_0 + \sin(\alpha -\theta_0))}
{(K^2 - 2K \cos \alpha + 1)^2}
\end{equation} 
%
The solutions to this equation are given by \dots
\begin{enumerate}
\item[a.] Solving the equation $K\cos \theta_0 - \cos(\alpha - \theta_0) = 0$ yields
%
\begin{align*}
\alpha_1 & = \theta_0 - \arccos(K \cos \theta_0) \\
\alpha_2 & = \theta_0 + \arccos(K \cos \theta_0)
\end{align*}
%
\item[b.] Solving the equation $K\sin \theta_0 + \sin(\alpha -\theta_0) = 0$ yields
%
\begin{align*}
\alpha_3 & = \theta_0 - \arcsin(K \sin \theta_0) \\
\alpha_4 & = \pi + \theta_0 + \arcsin(K \sin \theta_0)
\end{align*}
%
\end{enumerate}
These solutions are valid for \dots




\section{Distance function in Polar Coordinates}
We can simplify the analysis by rewriting Eqn. \ref{eq:YDist} in polar coordinates $(R,\theta)$. We do this by setting $R_0^2 = X_0^2 + Y_0^2,$ and $\theta_0 = \arctan(Y_0/X_0)$. This yields 
%
\begin{equation}
D_0^2 = R_0^2 + (1 + K^2) t^2 V^2 - 
 2 V t(K V t \cos \alpha  - R_0 \cos(\alpha - \theta_0) + K R_0 \cos \theta_0) 
 \label{eq:Dpolar}
\end{equation}
%
To find the time at which Eqn. \ref{eq:Dpolar} is minimized, we find the roots of the derivative of Eqn. \ref{eq:Dpolar} with respect to $t$ which yields
%
\begin{equation}
t_{min} = \ds \frac{R_0}{V} \frac{\left[K \cos \theta_0 - \cos(\alpha - \theta_0)\right ]}{1- 2K \cos \alpha + K^2}
\label{eq:tmin}
\end{equation}
%
The above is negative when $K \cos \theta_0 < \cos(\alpha - \theta_0).$ Rewriting this gives the range of $\alpha$ for which the distance is solely increasing. 
%
\begin{equation}
\theta_0 - \arccos(K \cos \theta_0) < \alpha < \theta_0 + \arccos(K \cos \theta_0)
\label{eq:alphabound}
\end{equation}
%
For these values of $\alpha$ the minimum distance occurs at $t=0$ and is thus equal to the initial distance $R_0^2$. As there is no single best choice in escape angle, we call this a region of optimality. Note that the above also restricts $K$ to values less than or equal to unity.   

If we now substitute $t_{min}$ for $t$ in Eqn. \ref{eq:Dpolar}, we get the minimum distance as a function of $\alpha$ with parameters $K$ and $\theta_0$. Explicitly, this is given by the following 
%
\begin{equation}
\ol{D}^2_{min}= \ds\frac{{D}^2_{min}}{R_0^2 }=
\ds\frac{\left ( \sin(\alpha - \theta_0) + K \sin \theta_0 \right )^2}{K^2-2 K \cos \alpha +1} 
\label{eq:Dmin_polar}
\end{equation}
%
To find the escape angle which yields the largest minimum distance, we solve the following equation
%
\begin{equation}
0 = \frac{\d \ol{D}^2_{min}}{\d \alpha} = 
\frac{2(K \cos \alpha - 1)(K\cos \theta_0 - \cos(\alpha - \theta_0))(K\sin \theta_0 + \sin(\alpha -\theta_0))}
{(K^2 - 2K \cos \alpha + 1)^2}
\end{equation} 
%
The solutions to this equation are given by \dots
\begin{enumerate}
\item[a.] Solving the equation $K\cos \theta_0 - \cos(\alpha - \theta_0) = 0$ yields
%
\begin{align*}
\alpha_1 & = \theta_0 - \arccos(K \cos \theta_0) \\
\alpha_2 & = \theta_0 + \arccos(K \cos \theta_0)
\end{align*}
%
\item[b.] Solving the equation $K\sin \theta_0 + \sin(\alpha -\theta_0) = 0$ yields
%
\begin{align*}
\alpha_3 & = \theta_0 - \arcsin(K \sin \theta_0) \\
\alpha_4 & = \pi + \theta_0 + \arcsin(K \sin \theta_0)
\end{align*}
%
\item[c.] Solving the equation $K \cos \alpha - 1 = 0$ yields
\begin{equation*}
\alpha_5 = \arccos K^{-1}
\end{equation*}
\end{enumerate}
The first two solutions are valid for $K \leq 1$ and the last solution is valid for $K \geq 1$.















\end{document}